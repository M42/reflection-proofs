\documentclass[a4paper]{article}

\usepackage{a4wide}


\usepackage{hyperref}
\usepackage{url}


\author{Paul van der Walt\footnote{\url{paul@denknerd.org}}}
\date{\today}
\title{Exploring Agda's New Reflection API}
\begin{document}

\maketitle


\section{Preamble}

This document is intended to give a brief overview of the progress so far concerning
this master's thesis project, as well as give a hint as to the direction we plan to follow
for the remaining part of the project. The Introduction section gives some background and
overview regarding the subject matter, and the Contributions section gives an overview of
the tangible contributions made, including planned and future (as yet unimplemented) contributions.



\section{Introduction}


The aim of this project is to explore the new (since version 2.2.8) reflection API included in the Agda
compiler, and come up with a number of neat examples of its applications. Reflection in programming language
terms means that one can convert concrete syntax of the language (terms and expressions, etc.) into an abstract
representation (usually some parsed tree; this process is often referred to as ``quoting'')
which can be traversed, inspected and modified using the same programming
language. Usually (but not always) languages supporting reflection have the possibility to
``unquote'' again, effectively
converting the abstract representation of the expression back into something the compiler
can turn into runnable code. Using this technique is often referred to as metaprogramming,
since one can inspect (``reflect'') on the code of the program at run-time, and alter behaviour
or instantiate specific functionality (possibly from more general code) at run-time.


This technique provides many interesting possibilities for easing a programmer's life, and in
this project a number of such options will be explored and implemented as illustrations.



The first example that comes to mind
is the suggestively named technique of proof by reflection. The idea here is that instead of producing a specific
proof derivation tree for each specific instance (some true statement which can be produced using the rules), one
defines a ``proof recipe'' which can be invoked for any specific instance. This is easily illustrated with properties
such as the evenness of natural numbers, or the fact that certain boolean expressions are tautologies.

Another example is allowing conversion of well-typed STLC (simply typed lambda calculus) terms to be quoted and
then automatically converted to SKI-combinators or CPS (continuation passing style). This is both an illustration of
the powers of dependently typed programming (since one can define a datatype such that only well-typed lambda
functions can be expressed) as a nice example of what can be done with a powerful reflection mechanism.

Generic programming is another technique which stands to benefit from good reflective capabilities, since even
though using an intermediate generic representation type for datatypes allows the definition of generic functions
which aid reuse of code, often embedding-projection functions still need to be defined between this representation type
and all the instances of datatypes one wants to be able to use with the generic functions. If one only needs to
pass the (quoted) datatype definition to a function which generates an EP-pair of functions which can be unquoted,
this process could be completely automated. Often defining such EP-pairs is a rather mechanical and tedious process
anyway.

Finally certain design patterns could possibly be abstracted away using reflection. Take the Bove-Capretta method
as an example, where non-terminating functions (at least, those that doterminate but that don't pass the
termination checker because of a complex recursion structure) can be rewritten to include an ``auxiliary'' argument
which does decrease structurally on each recursive call; this could possibly be generated at compile-time, so
the programmer needn't do the rewriting manually, only the proof obligation that would otherwise be the result of the
BC method would need to be given.


\section{Contributions}


This project has contributed a number of tangible results to the Agda ecosystem. The first of which, which is possibly
the most useful but least academically interesting, is a user-friendly manual to using the Reflection API. The Agda
release notes do include a short summary of the datatypes and functions involved, but in practise it takes quite some
frustrating hours to familiarise oneself with the way the AST is put together. This thesis contains a section on
using the reflection API, along with a few simple, practical examples, and a few tips and tricks for effectively
converting between the Agda Term type and your own (restrictive, property-ensuring) AST.

There are also a number of examples showing how the reflection API can be put to good use, for example the 
technique of proof by reflection





\end{document}
