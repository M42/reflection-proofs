\documentclass[a4paper]{article}

\author{Paul van der Walt\footnote{\url{paul@denknerd.org}}}
\date{\today}
\title{Exploring Agda's New Reflection API}
\begin{document}

\maketitle


\section{Preamble}

This document is intended to give a brief overview of the progress so far concerning
this master's thesis project, as well as give a hint as to the direction we plan to follow
for the remaining part of the project. The Introduction section gives some background and
overview regarding the subject matter, and the Contributions section gives an overview of
the tangible contributions made, including planned and future (as yet unimplemented) contributions.



\section{Introduction}


The aim of this project is to explore the new (since version 2.2.8) reflection API included in the Agda
compiler, and come up with a number of neat examples of its applications. Reflection in programming language
terms means that one can convert concrete syntax of the language (terms and expressions, etc.) into an abstract
representation (usually some parsed tree; this process is often referred to as ``quoting'')
which can be traversed, inspected and modified using the same programming
language. Usually (but not always) languages supporting reflection have the possibility to
``unquote'' again, effectively
converting the abstract representation of the expression back into something the compiler
can turn into runnable code. Using this technique is often referred to as metaprogramming,
since one can inspect (``reflect'') on the code of the program at run-time, and alter behaviour
or instantiate specific functionality (possibly from more general code) at run-time.


This technique provides many interesting possibilities for easing a programmer's life, and in
this project a number of such options will be explored and implemented as illustrations.



The first example that comes to mind
is the suggestively named technique of proof by reflection. The idea here is that instead of producing a specific
proof derivation tree for each specific instance (some true statement which can be produced using the rules), one
defines a ``proof recipe'' which can be invoked for any specific instance. This is easily illustrated with properties
such as the evenness of natural numbers, or the fact that certain boolean expressions are tautologies.

Another 


\section{Contributions}






\end{document}
